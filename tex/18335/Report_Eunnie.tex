\documentclass[11pt,letterpaper]{article}
\usepackage[utf8]{inputenc}
\usepackage{fontenc,multicol,amsmath,amssymb,comment,graphicx}
\usepackage{natbib}
\usepackage{subfig,siunitx,setspace,xcolor}

\addtolength{\hoffset}{-2cm}
\addtolength{\textwidth}{4cm}
\addtolength{\voffset}{-2.5cm}
\addtolength{\textheight}{4cm}

%----------commands----------------------------------%
\newcommand{\diff}{\mathrm{d}}
\newcommand{\p}{\partial}
\newcommand{\Qmat}{\underline{\underline{\mathbf{Q}}}}
\newcommand{\Mmat}{\underline{\underline{\mathbf{M}}}}
\newcommand{\Cvec}{\underline{\mathbf{C}}}
\newcommand{\Fvec}{\underline{\mathbf{F}}}
%------------------------------------------------------%
\author{Yoon Kyung (Eunnie) Lee \\eunnie12@mit.edu}
\title{\textbf{Nonlinear Eigenproblem for Optical Resonance}\\18.335 Term Project Report}
\date{\today}
\begin{document}
\maketitle
%%%%%%%%%%%%%%%%%%%%%%%%%%%%%%%%%%%%%%%%%%%%%%%%%%%%%%%
\abstract{
This project aims to implement and test a nonlinear eigenproblem by using two different numerical root-finding algorithms. The first is the conventional Newton's method, and the second is a contour-integral-based method developed by Beyn.\citep{beyn_integral_2012} Combination of the two methods will be designed, implemented, and tested. }

\begin{multicols}{2}
Follow JSIAM Letters publication form. 
reference Yokota et al. 

\section{Introduction}
We propose a solution method for nonlinear eigenvalue problems, (NEPs) hybridizing Beyn's contour integral method\citep{beyn_integral_2012} with Newton's root-finding method. Contour integral method does not require an initial guess, but is computationally expensive because a matrix inverse needs to be computed at every point on the contour quadrature. On the other hand, Newton's method is computationally less expensive but depends on the quality of the initial guess provided. our approach is to use these two methods together. The result from Beyn's method is used as the initial guess for Newton's method. 

A general NEP involves finding eigenpairs $(\omega,\mathbf{v})$ that satisfy
\begin{equation}\label{eq:Eig}
    A(\omega) \mathbf{v} = 0,
\end{equation}
where $A(\omega)$ is a matrix-valued function that is analytic, $\omega$ is the nonlinear eigenvalue, and $\mathbf{v}$ is the corresponding eigenvector.

\section{Background}

General solution method for eigenproblems


\subsection{Spectral decomposition of $A^{-1}(z)$}
Finding an eigenvalue is equivalent to finding singular points of the matrix problem $A(\omega)$. This is equivalent to searching for the poles of the inverse, $A(\omega)^{-1}$. The poles of the inverse problem are equivalent to the eigenvalues of the original problem.
\begin{equation}\label{Keldish}
A^{-1}(z) = \sum\limits_{i=1}^{k}\frac{P_i}{z-\omega_i}+R(z)
\end{equation}
where $P_i$ are spectral projection with respect to $\omega_i$, $k$ is the number of eigenvalues inside the given contour $\Gamma$ 
\subsection*{Test Case 1. Polynomial Eigenproblem}
For simplicity, we first construct a polynomial eigenproblem in the following form:
\begin{equation}\label{eq:polyeig}
A(\omega)=\sum\limits_{i=0}^{p}\omega^iA_i.
\end{equation}
All numerical implementations are tested using this test case. 
\subsection*{Test Case 2. Optical Eigenproblem from Boundary Element Method}
The eigenvalue problem of scientific interest is to find the resonance frequencies of metallic particles. This is a nonlinear eigenproblem because optical resonance of metallic objects is governed by the collective motion of the electrons, which has a nonlinear dependence on the angular frequency $\omega$. 
In this case, $A(\omega)$ is constructed by discretizing the integral form of the Maxwell's equations using the Boundary Element Method (BEM). The BEM matrix $A\mathbf{v}=\mathbf{f}$ relates the incident field vector $\mathbf{f}$ with the induced surface current vector $\mathbf{v}$, expressed using piecewise basis vectors. 

The Eigenvalues are computed by searching for the root of the following function:
\begin{equation}\label{eq:det}
f(\omega)=\text{det}\left(A(\omega)\right) = 0.
\end{equation}  

\section{Algorithm Description}
\subsection*{Newton's Method for Nonlinear Eigenproblems}
Numerical algorithms for nonlinear eigenvalue problems are usually based on Newton's method or are extensions of techniques for the standard eigenvalue problem. The Newton's Method finds the root of eq. \ref{eq:det} by iteratively performing the following.
\begin{equation}
\omega \rightarrow \omega - \frac{f(\omega)}{f'(\omega)} = \omega - \frac{\mathrm{det}(A(\omega))}{ \mathrm{tr} \left( \mathrm{adj}(A(\omega))\,\frac{dA(\omega)}{d\omega} \right)}
= \omega - \frac{1}{ \mathrm{tr} \left( A^{-1}(\omega)\,\frac{dA(\omega)}{d\omega} \right)}
\end{equation}
The quality of the initial guess makes a big difference for the performance of Newton's method. 
\subsection*{Beyn's Contour Integral-based Method for Nonlinear Eigenproblems}
On the other hand, the contour integral method is different from the Newton's method or other variants of the standard eigenvalue problem. THis method reduces the nonlinear eigenvalue problem into a k-dimensional linear problem by evaluating a contour integral around $\Gamma$ that surrounds $k$ eigenvalues in the complex plane:
\begin{equation}\label{eq:cont}
\frac{1}{2\pi i}\int\limits_{\Gamma}\omega^p A^{-1}(\omega)\hat{V}d\omega,
\end{equation}
where $p=0,1$ and $\hat{V} \in \mathbb{C}^{m,k}$ is generally taken as a random matrix. It is assumed that every eigenvalue $\lambda$ is isolated, and matrix inverse $A^{-1}(\omega)$ is meromorphic at $\lambda$, meaning that there exist $\kappa\in\mathbb{N}$ and $S_j\in \mathbb{C}^{m,m}$ for $j\geq -\kappa$ such that $S_{-\kappa}\neq 0$, and 
\begin{equation}\label{eq:Ainv}
A^{-1}(\omega)=\sum\limits_{j=-\kappa}^{\infty} S_j (\omega-\lambda)^j,\;\;
\;\; \omega \in \mathcal{U} \setminus \lbrace\lambda\rbrace 
\end{equation}
for some neighborhood $\mathcal{U}$ of $\lambda$. The theorem of Keldysh\citep{keldysh1951characteristic} is used to represent the singular part of eq. \ref{eq:Ainv} in terms of eigenvectors of $A$ and $\mathrm{Adj}(A)$. This method resembles the search of poles in the complex plane using contour integrals. 

The contour integrals in eq. \ref{eq:cont} are calculated approximately by trapezoidal sum, and therefore the main numerical effort for $N$ quadrature points arise from solving $N$ LU-decompositions and $Nk$ linear systems. While this operation count for a fine-meshed contour integral is a disadvantage, a big advantage is that no initial approximation is needed for the eigenvalues and eigenvectors. Therefore it is expected that an optimal performance will be achieved if Beyn's method is combined with Newton's method. 


\section{Numerical Example}
* Language: The algorithms will be implemented in C/C++.\\
In reality in MATLAB. 
* Discretization of the PDE: The open-source numerical implementation SCUFF-EM.\citep{SCUFF1}\\
* Validation: 
- The numerical integration used in the Beyn's contour integral method will be validated using the evaluation of a known function. \\
- The resulting eigenvalues and eigenmodes will be tested against the eigenmodes computed by the rigorous multiple scattering theory for spherical particles. \citep{xu1995electromagnetic}\\
* Comparison: The performance will be tested and compared between Newton's method, Beyn's contour integral method, and their combinations.  \\
* How to Benchmark Algorithms: flop counts, convergence-rate comparisons\\
Note: Comments and recommendations would be appreciated, especially related to better methods to benchmark the algorithms.  

\section{Conclusion}
In conclusion, 
%%%%%%%%%%%%%%%%%%%%%%%%%%%%%%%%%%%%%%%%%%%%%%%%%%%%%%%%
%% References %%
%\bibliography{Proposal_Eunnie_ref}{}
\bibliography{Proposalref_20150320}
\bibliographystyle{plain}
\end{multicols}
\end{document}